\documentclass[12pt, letterpaper, french]{article}


\usepackage[french]{babel}
\usepackage[hidelinks]{hyperref}
\usepackage{tikz}
    \usetikzlibrary{patterns}
\usepackage{venndiagram} % pour les diagrammes de venn

\title{Rapport de projet d'Outil Pour le Calcul des Prédicats (OPCP)}
\author{Lucas Coussement, Alice Basone}
\date{31 décembre 2023}

\begin{document}

\maketitle
\renewcommand*\abstractname{Résumé}
\begin{abstract}
    Ce rapport a pour but d'exposer le travail qui a été produit dans le but
    du projet d'OPCP. Ce projet concerne la programmation d'un outil permettant
    de tester la compatibilité entre des prémisses et une conclusion à l'aide de
    diagrammes de Venn.
\end{abstract}

\newpage
\tableofcontents
\newpage
\section{Introduction}
\subsection{Diagramme de Venn}
Les diagrammes de Venn sont des représentations graphiques de relations entre 
ensembles. Ils sont composés de cercles qui représentent les ensembles et de 
zones qui représentent les relations entre ces ensembles. Ces zones peuvent être
vides ou non pour illustrer si l'ensemble correspondant à la zone est vide ou 
non.

\begin{figure}[h]
    \centering
    \begin{venndiagram2sets}
    \end{venndiagram2sets}
    \caption{Exemple de diagramme de Venn}
\end{figure}

Dans l'exemple de la figure 1, le cercle de gauche représente l'ensemble $A$ et 
le cercle de droite représente l'ensemble $B$. La zone commune aux deux cercles 
représente l'intersection des deux ensembles, c'est-à-dire l'ensemble $A \cap B$.

Il est possible de représenter des relations entre autant d'ensemble que l'on 
souhaite même si en pratique, on se limite à 3 ou 4 ensembles.

\begin{figure}[h]
    \centering
    \begin{venndiagram3sets}
    \end{venndiagram3sets}
    \caption{Exemple de diagramme de Venn à 3 ensembles}
\end{figure}
\newpage

En remplissant les zones d'un diagramme on indique que la partie coloriée de
l'ensemble est remplie. Par exemple, si on colorie la zone commune aux trois 
cercles de la figure 2, on indique que l'intersection des trois ensembles est 
non-vide, autrement dit, que $A \cap B \cap C \not = \emptyset $.

\begin{figure}[h]
    \centering
    \begin{venndiagram3sets}[shade=red]
        \fillACapBCapC
    \end{venndiagram3sets}
    \caption{Exemple de diagramme de Venn à 3 ensembles avec l'intersection des zones coloriée}
\end{figure}
\subsection{Diagramme de Venn et Logique propositionnelle}

Il est possible d'utiliser les diagrammes de Venn pour représenter des formules 
de la logique propositionnelle. Pour cela, on utilise un diagramme de Venn pour 
chaque atome de la formule. On colorie ensuite les zones correspondant aux 
valeurs de vérité de chaque atome. On peut ensuite déterminer la valeur de
vérité de manière visuelle en regardant les zones coloriées du diagramme.

\begin{figure}[h]
    \centering 
    \begin{venndiagram2sets}[shade=red]
        \fillACapB 
    \end{venndiagram2sets}
    \caption{Exemple de diagramme de Venn pour la formule $A \land B$}
\end{figure}

\subsection{Diagramme de Venn et prédicats}

Il est également possible d'utiliser les diagrammes de Venn pour représenter 
des formules avec quantificateur. Pour cela, on utilise un diagramme de Venn
de la manière suivante :
\begin{itemize}
    \item Lorsque la formule est de la forme $\forall x \alpha$, on hachure les 
    zones qui ne sont pas coloriées par le diagramme de Venn de $\alpha$.	
    \item Lorsque la formule est de la forme $\exists x \alpha$, on créer autant 
    de diagrammes de Venn que de valeurs possibles pour $x$, c'est à dire qu'on 
    créer un diagramme pour chaque zone coloriée par le diagramme de Venn de 
    $\alpha$. Les zones coloriées seront alors des croix.
\end{itemize}

\begin{figure}[h]
    \centering 
    \begin{venndiagram2sets}[shade={}]
        \begin{scope}[every path/.append style={pattern=north east lines}]
            \fillANotB
        \end{scope}
    \end{venndiagram2sets}
    \caption{Exemple de diagramme de Venn pour la formule $\forall x (A(x) \to B(x))$}
\end{figure}

\begin{figure}[h]
    \centering 
    \begin{venndiagram2sets}[shade={}, 
        labelOnlyA={x}]
        \begin{scope}[every path/.append style={pattern=north east lines}]
        \end{scope}
    \end{venndiagram2sets}
    \begin{venndiagram2sets}[shade={}, 
        labelAB={x}]
        \begin{scope}[every path/.append style={pattern=north east lines}]
        \end{scope}
    \end{venndiagram2sets}
    \caption{Exemple des diagramme de Venn pour la formule $\exists x A(x)$}
\end{figure}
\newpage
\section{Implémentation des formules de la logique propositionnelle}

Les formules de la logique propositionnelle sont représentées en OCaml par un
type somme \emph{formule\_log\_prop} qui contient les différents constructeurs
de formules de la logique propositionnelle.

\subsection{Table de vérité}

La fonction qui donne la table de vérité d'une formule de la logique
propositionnelle est importante pour la suite du projet car elle permettra de 
créer les diagrammes de Venn en fonction des valeurs de vérité des atomes de la 
formule. Cette fonction prend en argument \emph{alpha} qui est une liste 
représentant les atomes présents dans la formule et \emph{f} qui est la formule
de la logique propositionnelle. 

\subsection{Comportement de Table de vérité}

\emph{table\_verite} est une fonction récursive qui se base sur le principe de 
la création d'un arbre de toutes les combinaisons possible de valeurs de vérité
pour chaque atome. Afin de tester toutes les combinaisons possibles, on utilise
une fonction auxiliaire qui prend en argument : 
\begin{itemize}
    \item une liste des atomes existants non-traité, 
    \item la formule,
    \item une liste des atomes qui sont actuellement vrai 
    \item une liste des résultats de la table de vérité.
\end{itemize}
\begin{figure}[h]
    \centering
    \begin{tikzpicture}[level distance=1.5cm,
    level 1/.style={sibling distance=3cm},
    level 2/.style={sibling distance=1.5cm}]
    \node { [ ] }
        child {node { [A]}
        child {node { [A, B] }}
        child {node {[A]}}
        }
        child {node { [ ] }
        child {node {[B]}}
        child {node {[ ]}}
        };
    \end{tikzpicture}
    \caption{Exemple d'arbre de combinaisons de valeurs de vérité pour les atomes A et B}
\end{figure}

Cet arbre est créé récursivement en faisant deux appels récursifs à la fonction 
auxiliaire. Le premier appel récursif ajoute l'atome à la liste des atomes 
actuellement vrai et le second appel récursif n'ajoute pas l'atome à la liste.
Ainsi en évaluant la formule sur toutes les feuilles de l'arbre on obtient la 
table de vérité de la formule.

\subsection{Fonctions pratiques}

Certaines fonctions ont été rajoutées afin de faciliter la création de la 
fonction \emph{table\_verite}. Ces fonctions sont les suivantes : 

\begin{itemize}
    \item \emph{atome\_dans\_f} : Cette fonction permet de lister tous les atomes présents dans une formule.
    \item \emph{eval\_Formule\_Log\_Prop} : Cette fonction permet d'évaluer une
    formule de la logique propositionnelle en fonction d'une liste d'atomes (tous les atomes présents dans la liste sont considéré comme vrai).
\end{itemize}

\section {Implémentation des formules pour les syllogismes étendus}

Les formules pour la représentation des syllogismes étendus sont de la forme :
\begin{itemize}
    \item $\forall x \alpha$ : \emph{PourTout} $\alpha$
    \item $\exists x \alpha$ : \emph{IlExiste} $\alpha$
\end{itemize}

Avec $\alpha$ une formule de la logique propositionnelle.
Dans cette partie, nous n'aurons qu'à définir une fonction qui permet de les 
transformer en chaîne de caractère afin de pouvoir les afficher.

\section{Implémentation des diagrammes de Venn}

Un diagramme de Venn est représenté par un tableau associatif qui associe à 
chaque ensemble de prédicat (chaque zone du diagramme) une valeur "Vide" ou "NonVide". Le vide correspond à une zone hachuré et le non-vide à la croix dans la zone \footnote{Voir les figures 5 et 6}.

Les clés sont donc des ensembles de chaine de caractères et les valeurs des types somme.

\subsection{Réponse à la Question 1} 

\emph{Énoncé :} Expliquer comment calculer les diagrammes de Venn associés à
une valeur $f$ de type $formule\_syllogisme$ en fonction du quantificateur 
et de la table de vérité de la valeur $f'$ de type $formule\_log\_prop$
définissant $f$.
\\

Pour calculer les diagrammes de Venn associés à une valeur $f$ de type 
$formule\_syllogisme$ nous devons traîter deux cas. Nous savons qu'une formule 
de type $formule\_syllogisme$ est soit de la forme \emph{PourTout} $f'$ soit de
la forme \emph{IlExiste} $f'$ avec $f'$ une formule de type $formule\_log\_prop$.
\\ 

Dans le cas où la formule est de la forme \emph{PourTout} $f'$, nous devons calculer la table de vérité de $f'$ et ne garder que les interprétation où $f'$ est faux. Notre objectif étant de trouver toutes les zones du diagramme à hachuré car nous saurons que pour toutes les valeurs de $x$ de l'ensemble de prédicat, $f'$ sera faux dans ces zones. 
\\
\\
\emph{Exemple : dans la situation où $f = \forall x (A(x) \land B(x))$.}
La table de vérité de la formule $f' = A \land B$  est la suivante :
\begin{verbatim}
    (["a"; "b"], true); (["b"] , false); (["a"], false); ([], false)
\end{verbatim}
Nous ne gardons que les interprétations où $f'$ est faux, c'est-à-dire :
\begin{verbatim}
    (["b"] , false); (["a"], false); ([], false)
\end{verbatim}
Ces zones seront donc celles qui seront hachurées dans le diagramme de Venn.

\begin{figure}[h]
    \centering
    \begin{venndiagram2sets}[shade={}]
        \begin{scope}[every path/.append style={pattern=north east lines}]
            \fillANotB
            \fillNotA
        \end{scope}
    \end{venndiagram2sets}
    \caption{Exemple de diagramme de Venn pour la formule $\forall x (A(x) \land B(x))$}
\end{figure}

Ce qui nous donne le diagramme 
\begin{verbatim}
    [["{a}", Vide ; "{b}", Vide ; "\emptyset", Vide]]
\end{verbatim}

Dans le cas où la formule est de la forme \emph{IlExiste} $f'$, nous devons calculer la table de vérité de $f'$ et ne garder que les interprétations où $f'$
est vrai. Cela nous permettra de créer les diagrammes de Venn correspondant à toutes les valeurs possible de $x$ pour lesquelles $f'$ peut être vrai.
\\
\\

\emph{Exemple : dans la situation où $f = \exists x A(x)$.} La table de vérité
de la formule $f' = A$ est la suivante : 
\begin{verbatim}
    (["a"], true); (["b"], false); (["a"; "b"], true); ([], false)
\end{verbatim}

Nous ne gardons que les interprétations où $f'$ est vrai, c'est-à-dire :
\begin{verbatim}
    (["a"], true); (["a"; "b"], true)
\end{verbatim}

Il y aura autant de diagrammes que d'interprétations correctes de $f'$, c'est-à-dire deux dans notre cas. 

\begin{figure}[h]
    \centering
    \begin{venndiagram2sets}[shade={},
        labelOnlyA={x}]
        \begin{scope}[every path/.append style={pattern=north east lines}]
        \end{scope}
    \end{venndiagram2sets}
    \begin{venndiagram2sets}[shade={},
        labelAB={x}]
        \begin{scope}[every path/.append style={pattern=north east lines}]
        \end{scope}
    \end{venndiagram2sets}
    \caption{Exemple de diagramme de Venn pour la formule $\exists x A(x)$}
\end{figure}

Ce qui nous donne les diagrammes

\begin{verbatim}
    [["{a}", NonVide];
    ["{a, b}", NonVide]]
\end{verbatim}

\subsection{Implémentation de la fonction \emph{diag\_from\_formule}}
En suivant la réponse à la question 1, nous avons implémenté la fonction 
\emph{diag\_from\_formule} qui prend en argument la liste des noms de fonction 
que prend la formule, une formule de type \emph{formule\_syllogisme} et qui renvoie un diagramme de Venn correspondant à la formule.

\newpage
\subsection{Réponse à la Question 2}

\emph{Énoncé :} Comment se calcule la conjonction de deux diagrammes de Venn 
fonctionnels ? 
\\
\\

La conjonction de diagrammes consiste à créer un nouveau diagramme qui est 
une sorte d'union des deux diagrammes de Venn. Dans certains cas, il est 
impossible d'effectuer une conjonction de diagrammes. C'est le cas lorsque les 
diagrammes se contredisent. Par exemple, si un diagramme indique qu'une zone est
vide et que l'autre indique qu'elle est non-vide, il sera impossible de faire 
une conjonction de ces deux diagrammes.

On choisit donc un des deux diagrammes que nous souhaitons conjoncter et on 
cherche à savoir pour chaque élément : 
\begin{itemize}
    \item Si l'élément est présent dans les deux diagrammes, on vérifie si les 
    deux diagrammes indiquent que l'élément est vide ou non-vide. Si les deux 
    diagrammes indiquent la même chose, on ajoute l'élément au nouveau diagramme
    avec la même valeur que les deux diagrammes. Si les deux diagrammes 
    indiquent des valeurs différentes, on ne peut pas effectuer la conjonction et nous renvoyons None 
    \item Si l'élément n'est présent que dans un des deux diagrammes, on ajoute
    l'élément au nouveau diagramme avec la même valeur que le diagramme qui 
    contient l'élément.
\end{itemize}

A la fin si nous n'avons pas renvoyé None, nous ajoutons tous les éléments du deuxième diagramme qui n'étaient pas présent dans le diagramme choisit au nouveau diagramme et nous renvoyons le nouveau diagramme.

\subsection{Implémentation de la fonction \emph{conj\_diag}}
L'implémentation consiste à un \emph{Diag.fold} sur le premier diagramme en 
utilisant la fonction \emph{Diag.add} sur l'accumulateur lorsque l'élément est 
présent dans les deux diagrammes et qu'il n'est pas contradictoire ou qu'il est 
présent dans le premier diagramme. Si l'élément est contradictoire, on renvoie
None. A la fin, on ajoute tous les éléments du deuxième diagramme qui ne sont
pas présents dans le premier diagramme encore avec un \emph{Diag.fold}.

\subsection{Réponse à la Question 3}

\emph{Énoncé :} Expliquer comment déterminer la compatibilité de deux diagrammes en OCaml. Donner un exemple illustrant le fait que la compatibilité n'est pas symétrique. 
\\
\\

Une prémisse est compatible avec une conclusion lorsque : 
\begin{itemize}
    \item Tous les éléments de la conclusion sont présents dans la prémisse.
    \item Les valeurs de vérité des éléments de la conclusion sont les mêmes
    que celles des éléments de la prémisse.
\end{itemize}

En d'autres termes si nous parcourons le diagramme de conclusion avec un 
\emph{Diag.fold} et que pour chaque élément nous trouvons un élément dans le 
diagramme des prémisses avec la même valeur, alors les deux diagrammes sont 
compatibles. Sinon, ils ne le sont pas.

Cette relation n'est pas symétrique car si nous prenons l'exemple suivant : 

\begin{figure}[h]
    \centering
    \begin{venndiagram2sets}[shade={}]
        \begin{scope}[every path/.append style={pattern=north east lines}]
            \fillOnlyA
            \fillOnlyB
        \end{scope}
    \end{venndiagram2sets}
    \caption{Exemple de diagramme de Venn}
\end{figure}


\begin{figure}[h]
    \centering
    \begin{venndiagram2sets}[shade={}]
        \begin{scope}[every path/.append style={pattern=north east lines}]
            \fillOnlyA
        \end{scope}
    \end{venndiagram2sets}
    \caption{Exemple de diagramme de Venn}
\end{figure}

Ici si nous prenons le diagramme de la figure 10 comme prémisse et celui de la 
figure 11 comme conclusion, nous pouvons voir que la conclusion est compatible 
(elle est incluse dans la prémisse) avec la prémisse. A l'inverse, si nous prenons le diagramme
de la figure 11 comme prémisse et celui de la figure 10 comme conclusion, nous 
pouvons voir que la conclusion n'est pas compatible (elle n'est pas incluse dans la prémisse) avec la prémisse.

\subsection{Implémentation des fonctions de compatibilité}

La fonction de compatibilité entre diagramme (\emph{est\_compatible\_diag\_diag}) est implémentée en utilisant un \emph{Diag.fold} sur le diagramme de conclusion. A chaque élément du diagramme de conclusion, on vérifie si l'élément est présent dans le diagramme de prémisse et si les deux éléments ont la même valeur.

La fonction de compatibilité entre diagramme de prémisse et liste de diagramme de conclusion est implémentée en utilisant un \emph{List.exist} qui vérifie si
au minimum une prémisse est compatible avec un des diagrammes de conclusion.

La fonction de compatibilité entre liste de diagramme de prémisse et liste de 
diagramme de conclusion est implémentée en utilisant un \emph{List.for\_all} qui
vérifie si toutes les prémisse sont compatibles avec au moins un des diagrammes 
de conclusion.

Enfin, la fonction de compatibilité entre une liste de formule de prémisse et 
une formule de conclusion se fait en transformant toutes les formules en 
diagrammes de Venn, en combinant les formules de prémisses entre elles et en 
vérifiant si le résultat est compatible avec la conclusion.

\subsection{Combinaison}

La combinaison d'une liste de prémisse consiste à isoler toutes les prémisses 
qui sont de la forme "PourTout" et à les combiner entre elles en un seul 
diagramme. On isole ensuite toutes les prémisses qui sont de la forme "IlExiste"
et pour chaque prémisses de ce type, on combine le diagramme des prémisses 
"PourTout" avec chaque diagramme de la prémisses "IlExiste" et ce pour toutes 
les prémisses "IlExiste". On fini avec une liste de diagramme de Venn ne 
comportant qu'un seul "non-vide" par ensemble de prédicat. Lorsqu'un diagramme 
issu d'un "IlExiste" comporte un élément ayant pour valeur "non-vide" mais que cet élément est déjà présent dans le "PourTout" avec pour valeur "vide", on le supprime de la liste.

\begin{figure}[h]
    \centering
    \begin{venndiagram2sets}[shade={}]
        \begin{scope}[every path/.append style={pattern=north east lines}]
            \fillOnlyA
        \end{scope}
    \end{venndiagram2sets}
    \begin{venndiagram2sets}[shade={}, labelOnlyB={x}]
        \begin{scope}[every path/.append style={pattern=north east lines}]
        \end{scope}
    \end{venndiagram2sets}
    \begin{venndiagram2sets}[shade={}, labelAB={x}]
        \begin{scope}[every path/.append style={pattern=north east lines}]
        \end{scope}
    \end{venndiagram2sets}
    \begin{venndiagram2sets}[shade={}, labelOnlyA={x}]
        \begin{scope}[every path/.append style={pattern=north east lines}]
        \end{scope}
    \end{venndiagram2sets}
    \caption{Exemple d'une liste de prémisses}
\end{figure}
\begin{figure}[h]
    \centering
    \begin{venndiagram2sets}[shade={}, labelOnlyB={x}]
        \begin{scope}[every path/.append style={pattern=north east lines}]
            \fillOnlyA
        \end{scope}
    \end{venndiagram2sets}
    \begin{venndiagram2sets}[shade={}, labelAB={x}]
        \begin{scope}[every path/.append style={pattern=north east lines}]
            \fillOnlyA
        \end{scope}
    \end{venndiagram2sets}
    \caption{Résultat de la combinaison des prémisses de la figure 12}
\end{figure}

\subsection{Témoins d'incompatibilité}

Les fonctions de témoins d'incompatibilité servent à trouver les diagrammes des
prémisses qui sont incompatible avec la conclusion. La première fonction
\emph{temoin\_incompatibilite\_premisses\_conc\_opt} renvoie un diagramme option
qui est un des diagrammes de prémisse incompatible avec la conclusion s'il existe, None sinon.

La seconde fonction \emph{temoins\_incompatibilite\_premisses\_conc} renvoie 
simplement toutes les prémisses qui sont incompatible avec la conclusion.

Ces deux fonctions sont implémentées en exécutant les fonctions de compatibilité
étape par étape et en gardant en mémoire les diagrammes de prémisse qui ne sont 
pas compatible avec la conclusion ou en renvoyant simplement un diagramme ou None pour la première fonction.

\section{Test}

Cette partie du projet est dédiée à la programmation d'une fonction qui permet 
de tester toutes les fonctions évoquées précédemment. Cette fonction testera 
la validité d'une liste de prémisses par rapport à une conclusion. Elle 
affichera chaque étape du processus de vérification et affichera le résultat 
final.

\subsection{Implémentation} 

La fonction de test est une synthèse de toutes les fonctions précédentes. Elle 
utilise les fonctions de transformation en chaîne de caractère des formules et 
des diagrammes de Venn afin de pouvoir les afficher. Elle utilise également les
fonctions de compatibilité, de combinaison, et de témoins d'incompatibilité afin
de vérifier la validité des prémisses par rapport à la conclusion.

\end{document}
